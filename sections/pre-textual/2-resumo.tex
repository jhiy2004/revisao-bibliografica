% Resumo/palavras-chave (língua vernácula)------------------------------------------------------
\newpage
\thispagestyle{empty}
\begin{center}
    \uppercase{\textbf{Resumo}}
\end{center}
A visualização científica é um dos principais recursos didáticos em trabalhos físicos e matemáticos usados em apresentações de trabalhos científicos, em palestras
ou até mesmo em sala de aula, consiste em um forma simples de apresentar um fenômeno para uma audiência. Esse trabalho tem como objetivo desenvolver uma aplicação
voltada a criação de animações de visualização científica, mais especificamente criar animações para o modelo matemático de águas rasas.


\begin{center}
    \uppercase{\textbf{Palavras-chaves}}
\end{center}
Palavra, Palavra, Palavra, Palavra, Palavra, Palavra, Palavra.
