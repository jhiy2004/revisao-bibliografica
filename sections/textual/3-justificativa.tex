\section{Justificativa}

As equações de águas rasas (Shallow Water Equations - SWE) descrevem a dinâmica de
fluidos em regimes nos quais a dimensão vertical é muito menor que as dimensões horizontais,
sendo amplamente utilizadas em estudos e hidrodinâmica, modelagem de ondas e previsão de
inundações. A solução numérica destas equações exige métodos computacionais eficientes para
resolver sistemas de equações diferenciais parciais, gerando grandes volumes de dados espaciais
e temporais.

A visualização adequada desses resultados é essencial para a interpretação física e validação
dos modelos. No entanto, a maioria das abordagens atuais emprega ferramentas de visualização
científica tradicionais (como ParaView e VisIt), que, embora robustas, não exploram todo o
potencial de interatividade e desempenho gráfico oferecido por APIs gráficas modernas, como
o Vulkan.

Dessa forma, identifica-se uma oportunidade para desenvolver um sistema que integre a
solução numérica das SWE - utilizando o framework Devito para discretização por diferenças
finitas - a uma API gráfica moderna (Vulkan), possibilitando renderização interativa e análise
visual em tempo real dos resultados, favorecendo a compreensão e validação do modelo.
