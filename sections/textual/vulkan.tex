\section{Visualização científica}
Tecnologias de computação gráfica em tempo real foram amplamentes usadas em visualização
científica por décadas, sendo o OpenGL a escolha mais popular, que é uma biblioteca
gráfica \textit{open-source} cridada em 1992.

De acordo com \textcite{datoviz}, ainda que o OpenGL seja amplamente utilizado,
\textit{APIs} gráficas de mais baixo nível como Vulkan (Khronos), WebGPU(W3C),
Metal (Apple) e DirectX 12 (Microsoft) estão sendo cada vez mais usadas pela comunidade
científica.

O Vulkan constitui uma API gráfica de baixo nível que disponibiliza controle fino sobre
a GPU e acesso aos recursos gráficos contemporâneos. Esse maior grau de controle, porém,
implica que tarefas como a apresentação dos quadros e a sincronização entre CPU e GPU
devem ser tratadas explicitamente pelo programador.
