\section{Introdução}
As Equações de Navier–Stokes constituem a formulação fundamental para descrever o
movimento de fluidos viscosos, tendo sido desenvolvidas no século XIX e permanecendo
até hoje como base da mecânica dos fluidos. Em diversos cenários, entretanto, é possível
empregar aproximações que preservam o comportamento essencial do escoamento com
menor custo computacional. Entre essas aproximações destacam-se as Equações de Águas
Rasas (Shallow Water Equations – SWE), que assumem que o comprimento de onda é muito
maior que a profundidade do fluido, tornando-se um modelo amplamente utilizado na
simulação de tsunamis, marés e escoamentos costeiros.

A solução analítica dessas equações diferenciais parciais é, em geral, inviável, uma vez
que requer conhecimento prévio das funções exatas que governam o sistema. Por essa razão,
técnicas numéricas, como os métodos de diferenças finitas, são amplamente empregadas para
aproximar as derivadas parciais e calcular a evolução das variáveis físicas ao longo do
tempo. Nesse contexto, ferramentas como o Devito têm desempenhado papel crescente ao
permitir a geração automatizada e otimizada de códigos numéricos destinados à simulação
desses modelos.

Além dos cálculos das variáveis físicas ao longo do tempo, é necessário que sejam
aplicadas técnicas de visualização científica para a interpretação dos resultados.
Para esse fim existem ferramentas como o ParaView e VisIt, que são ferramentas para
visualização de dados científicos e usam OpenGL para renderização. Entretanto, trabalhos
mais recentes, como o \textcite{datoviz} vêm explorando a possibilidade usar APIs gráficas
mais modernas, como o Vulkan.

A revisão bibliográfica está organizada da seguinte forma:

\begin{enumerate}[label=(\Roman*)]
    \item apresenta-se o fundamento teórico das SWE e dos métodos de discretização;
    \item discute-se o framework Devito como ferramenta para implementação de métodos de diferenças finitas de alto desempenho;
    \item analisam-se técnicas e ferramentas de visualização científica, com ênfase no uso de APIs gráficas modernas como o Vulkan.
\end{enumerate}
