\section{Matemática}

\subsection{Equações de Navier Stokes}

Um fluido é uma substância (líquida ou gasosa) que se deforma continuamente ao sofrer qualquer tensão.
Os fluidos são caracterizados por alguns atributos, os principais são a viscosidade e compressibilidade. A viscosidade busca medir
o quão internamente resistente é o fluido ao movimento e ao escoamento, um exemplo de fluido com alta viscosidade é o mel que flui mais
devagar do que um fluido de baixa viscosidade como a água. A compressibilidade se refere a capacidade do fluido
alterar seu volume sob pressão, gases são compressíveis enquanto líquidos são considerados praticamente incompressíveis.

Os fluidos podem ser classificados em duas grandes categorias, os fluidos newtonianos e os não-newtonianos.
Um fluido newtoniano é um fluido onde cada componente de tensão cisalhante aplicada é linearmente proporcional a
taxa de deformação. Um fluido não-newtoniano é um fluido que não respeita essa proporção.

As equações de Navier-Stokes são equações diferenciais parciais que descrevem a dinâmica dos fluidos, foram formuladas a partir das
Leis de Newton do movimento.

Considerando escoamentos isotérmicos e incompressíveis, as equações na forma vetorial que modelam
esse escoamento são a equação da continuidade e a equação do momento.

\begin{itemize}
    \item Equação da continuidade
        \begin{equation}
            \label{eq:continuidade}
            \nabla \cdot \mathbf{u} = 0 
        \end{equation}
    \item Equação do momento
        \begin{equation}
            \label{eq:momento}
            \rho \left(\frac{\partial \mathbf{u} ^ T}{\partial t} + \nabla \cdot (\mathbf{u}\mathbf{u}^{T})\right) = \nabla \cdot \mathbf{\sigma} + \rho \mathbf{g}
        \end{equation}
\end{itemize}

Para escoamentos bidimensionais, $\mathbf{u} = \left[ u \ v \right]^T$ é o vetor de velocidades dependentes do espaço e do tempo, $u(\mathbf{x}, t)$ e $v(\mathbf{x}, t)$,
onde $\mathbf{x} = \left[ x \ y \right]^T$. O tensor tensão total é dado por 
\[
    \mathbf{\sigma} = \tau - pI
\]

onde $p$ é a pressão dada em função do espaço e do tempo $p(x, y, t)$, $I$ é a matriz identidade e 
\[
    \tau = 
    \begin{bmatrix}
        \tau^{xx} & \tau^{xy} \\
        \tau^{xy} & \tau^{yy} \\
    \end{bmatrix}
\]

Assim como a pressão, as tensões dentro da matriz também estão em função do espaço e do tempo.

A matriz das tensões $\tau$ para fluidos newtonianos é linearmente proporcional à matriz taxa de deformação. Logo,
considerando um escoamento incompressível, $\tau$ é escrita da seguinte forma:
\[
    \tau = 2 \eta_{s} \mathbf{S},
\]
onde $\eta_{s}$ é a viscosidade do fluido e $\mathbf{S}$ é a matriz simétrica de taxa de deformação, dada por
\[
    \mathbf{S} = \frac{1}{2}\big(\nabla \mathbf{u} + (\nabla \mathbf{u})^T\big).
\]

Dada a definição da equação em sua forma vetorial é possível derivar para a forma de coordenadas cartesianas. Isso
é feito por meio da expansão dos vetores e multiplicação de matrizes.

\begin{equation}
\label{eq:divu}
\nabla \cdot \mathbf{u} = 
\begin{bmatrix}
    \frac{\partial}{\partial x} & \frac{\partial}{\partial y}
\end{bmatrix}
\begin{bmatrix}
    u \\
    v \\
\end{bmatrix}
= \frac{\partial u}{\partial x} + \frac{\partial v}{\partial y} = 0
\end{equation}

\begin{equation}
\label{eq:udt}
\frac{\partial \mathbf{u} ^ T}{\partial t} =
\begin{bmatrix}
    \frac{\partial u}{\partial t} & \frac{\partial v}{\partial t}
\end{bmatrix}
\end{equation}

\begin{equation}
\label{eq:divuut}
\nabla \cdot (\mathbf{u}\mathbf{u}^{T}) =
\begin{bmatrix}
    \frac{\partial}{\partial x} & \frac{\partial}{\partial y}
\end{bmatrix}
\begin{bmatrix}
    uu & uv \\
    uv & vv \\
\end{bmatrix}
=
\begin{bmatrix}
    \frac{\partial (uu)}{\partial x} + \frac{\partial (uv)}{\partial y} &
    \frac{\partial (uv)}{\partial x} + \frac{\partial (vv)}{\partial y}
\end{bmatrix}
\end{equation}

\begin{equation}
\label{eq:divtwoeta}
\resizebox{\textwidth}{!}{$
\begin{aligned}
\nabla \cdot \left(2 \eta_S \mathbf{S} - p \mathbf{I}\right)
&= 
\begin{bmatrix}
\frac{\partial}{\partial x} & \frac{\partial}{\partial y}
\end{bmatrix}
\begin{bmatrix}
2 \eta_S \frac{\partial u}{\partial x} - p & \eta_S \left(\frac{\partial u}{\partial y} + \frac{\partial v}{\partial x}\right) \\
\eta_S \left(\frac{\partial u}{\partial y} + \frac{\partial v}{\partial x}\right) & 2 \eta_S \frac{\partial v}{\partial y} - p
\end{bmatrix} \\[6pt]
&=
\begin{bmatrix}
2 \eta_S \frac{\partial^2 u}{\partial x^2} - \frac{\partial p}{\partial x} + \eta_S \left(\frac{\partial^2 u}{\partial y^2} + \frac{\partial}{\partial y}\left(\frac{\partial v}{\partial x}\right)\right) &
\eta_S \left(\frac{\partial}{\partial x}\left(\frac{\partial u}{\partial y}\right) + \frac{\partial^2 v}{\partial x^2}\right) + 2 \eta_S \frac{\partial^2 v}{\partial y^2} - \frac{\partial p}{\partial y}
\end{bmatrix} \\[6pt]
&=
\begin{bmatrix}
2 \eta_S \frac{\partial^2 u}{\partial x^2} - \frac{\partial p}{\partial x} + \eta_S \left(\frac{\partial^2 u}{\partial y^2} - \frac{\partial^2 u}{\partial x^2}\right) &
\eta_S \left(- \frac{\partial^2 v}{\partial y^2} + \frac{\partial^2 v}{\partial x^2}\right) + 2 \eta_S \frac{\partial^2 v}{\partial y^2} - \frac{\partial p}{\partial y}
\end{bmatrix} \\[6pt]
&=
\begin{bmatrix}
- \frac{\partial p}{\partial x} + \eta_S \left(\frac{\partial^2 u}{\partial x^2} + \frac{\partial^2 u}{\partial y^2}\right) &
- \frac{\partial p}{\partial y} + \eta_S \left(\frac{\partial^2 v}{\partial x^2} + \frac{\partial^2 v}{\partial y^2}\right)
\end{bmatrix}
\end{aligned}
$}
\end{equation}

\begin{equation}
\label{eq:rhog}
\rho \mathbf{g} = \rho
\begin{bmatrix}
    g_x & g_y
\end{bmatrix}
\end{equation}

Substituindo as equações de \eqref{eq:udt} até \eqref{eq:rhog} na equação 2, obtemos um sistema de duas equações, sabendo que \eqref{eq:divu} foi usado para chegar na equação \eqref{eq:divtwoeta}, o sistema é composto por 3 equações.

\begin{equation}
\rho
\left(
    \begin{bmatrix}
        \frac{\partial u}{\partial t} & \frac{\partial v}{\partial t}
    \end{bmatrix}
    +
    \begin{bmatrix}
        \frac{\partial (uu)}{\partial x} + \frac{\partial (uv)}{\partial y} &
        \frac{\partial (uv)}{\partial x} + \frac{\partial (vv)}{\partial y}
    \end{bmatrix}
\right)
= 
\rho
\begin{bmatrix}
    \frac{\partial u}{\partial t} + \frac{\partial (uu)}{\partial x} + \frac{\partial (uv)}{\partial y} &
    \frac{\partial v}{\partial t} + \frac{\partial (uv)}{\partial x} + \frac{\partial (vv)}{\partial y}
\end{bmatrix}
\end{equation}

\begin{equation}
    \nabla \cdot \mathbf{\sigma} + \rho \mathbf{g} =
\begin{bmatrix}
    - \frac{\partial p}{\partial x} + \eta_S \left(\frac{\partial^2 u}{\partial x^2} + \frac{\partial^2 u}{\partial y^2}\right) &
    - \frac{\partial p}{\partial y} + \eta_S \left(\frac{\partial^2 v}{\partial x^2} + \frac{\partial^2 v}{\partial y^2}\right)
\end{bmatrix}
+
\rho
\begin{bmatrix}
    g_x & g_y
\end{bmatrix}
\end{equation}

\begin{equation}
\left\{
\begin{aligned}
\rho
\Big(
    \frac{\partial u}{\partial t} + \frac{\partial (uu)}{\partial x} + \frac{\partial (uv)}{\partial y}
\Big)
&= \rho g_x - \frac{\partial p}{\partial x} + \eta_S \Big(\frac{\partial^2 u}{\partial x^2} + \frac{\partial^2 u}{\partial y^2}\Big), \\
\rho
\Big(
    \frac{\partial v}{\partial t} + \frac{\partial (uv)}{\partial x} + \frac{\partial (vv)}{\partial y}
\Big)
&= \rho g_y - \frac{\partial p}{\partial y} + \eta_S \Big(\frac{\partial^2 v}{\partial x^2} + \frac{\partial^2 v}{\partial y^2}\Big), \\
\frac{\partial u}{\partial x} + \frac{\partial v}{\partial y} 
&= 0
\end{aligned}
\right.
\end{equation}

\subsection{Equações de águas rasas}

As equações de águas rasas são um sistema de equações diferenciais parciais que descrevem o movimento dos fluidos onde
a profundidade é muito menor do que a extensão horizontal do fluido, muito usadas para modelar fenômenos como ondas, correntes
oceânicas e tsunamis.

Essas equações são derivadas a partir das equações de Navier-Stokes, mais especificamente uma variação das equações de Navier-Stokes
chamada de equações de Euler, onde são feitas algumas aproximações para simplificar o modelo matemático. Nessas equações o fluido
é assumido como isotérmico e invíscido. Invíscido é um termo usado para denominar um fluido idealizada com viscosidade zero.

\begin{equation}
\label{eq:swe-sis-1}
\left\{
\begin{aligned}
\frac{\partial \rho}{\partial t}
&+ \frac{\partial (\rho u)}{\partial x}
+ \frac{\partial (\rho v)}{\partial y}
+ \frac{\partial (\rho w)}{\partial z}
= 0,
\\[6pt]
\frac{\partial (\rho u)}{\partial t}
&+ \frac{\partial (\rho u^2 + p)}{\partial x}
+ \frac{\partial (\rho u v)}{\partial y}
+ \frac{\partial (\rho u w)}{\partial z}
= 0,
\\[6pt]
\frac{\partial (\rho v)}{\partial t}
&+ \frac{\partial (\rho u v)}{\partial x}
+ \frac{\partial (\rho v^2 + p)}{\partial y}
+ \frac{\partial (\rho v w)}{\partial z}
= 0,
\\[6pt]
\frac{\partial (\rho w)}{\partial t}
&+ \frac{\partial (\rho u w)}{\partial x}
+ \frac{\partial (\rho v w)}{\partial y}
+ \frac{\partial (\rho w^2 + p)}{\partial z}
= -\rho g.
\end{aligned}
\right.
\end{equation}

Onde $\rho$ é a densidade do fluido , $p$ é a pressão e $g$ é a constante gravitacional, e $u, v, w$ são as velocidades em cada
uma das direções x, y, z.

Assumindo que estamos lidando com um fluido incompressível, podemos tomar $\rho$ como uma constante, simplificando a equação.

\begin{equation}
\label{eq:swe-sis-2}
\left\{
\begin{aligned}
\frac{\partial }{\partial t}
&+ \frac{\partial u}{\partial x}
+ \frac{\partial v}{\partial y}
+ \frac{\partial w}{\partial z}
= 0,
\\[6pt]
\frac{\partial u}{\partial t}
&+ \frac{\partial ( u^2 + p)}{\partial x}
+ \frac{\partial (u v)}{\partial y}
+ \frac{\partial (u w)}{\partial z}
= 0,
\\[6pt]
\frac{\partial v}{\partial t}
&+ \frac{\partial u v}{\partial x}
+ \frac{\partial (v^2 + p)}{\partial y}
+ \frac{\partial (v w)}{\partial z}
= 0,
\\[6pt]
\frac{\partial w}{\partial t}
&+ \frac{\partial u w}{\partial x}
+ \frac{\partial v w}{\partial y}
+ \frac{\partial (w^2 + p)}{\partial z}
= -g.
\end{aligned}
\right.
\end{equation}

A principal simplificação que será feita na equação acima é assumir que o tamanho vertical da onda é muito menor do que o seu
tamanho horizontal, o que implica que a aceleração vertical é insignificante quando comparada com $g$. $\frac{Dw}{Dt} = 0$.

\begin{equation}
\label{eq:swe-sis-3}
\left\{
\begin{aligned}
\frac{\partial }{\partial t}
&+ \frac{\partial u}{\partial x}
+ \frac{\partial v}{\partial y}
+ \frac{\partial w}{\partial z}
= 0,
\\[6pt]
\frac{\partial u}{\partial t}
&+ \frac{\partial (u^2 + p)}{\partial x}
+ \frac{\partial (u v)}{\partial y}
+ \frac{\partial (u w)}{\partial z}
= 0,
\\[6pt]
\frac{\partial v}{\partial t}
&+ \frac{\partial (u v)}{\partial x}
+ \frac{\partial (v^2 + p)}{\partial y}
+ \frac{\partial (v w)}{\partial z}
= 0,
\\[6pt]
\frac{\partial p}{\partial z}
&= -g.
\end{aligned}
\right.
\end{equation}

Integrando a quarta equação do sistema \eqref{eq:swe-sis-3} em ambos os lados na variável $z$.

\begin{equation}
p = -gz + C
\end{equation}

\begin{figure}[H]
    \centering
    \caption{Esquema visual das equações de águas rasas}
    \includegraphics[width=0.5\linewidth]{images/shallow_water_sketch.png}\\
    \fonteimagem{Fonte: Devito}
    \label{fig:figura-swe}
\end{figure}

Considerando a \ref{fig:figura-swe} podemos simplificar um pouco as equações de águas rasas integrando as
variáveis $u$ e $v$ na variável $z$ e chamando de $M$ e $N$ respectivamente.

\begin{equation}
    \label{eq:swe-m}
    M =
    \int_{-h}^{\eta}
    u (x, y, z, t)
    \,dz =
    u (x, y, \tilde{z}, t)
    \int_{-h}^{\eta}
    \,dz =
    u (x, y, \tilde{z}, t)
    (\eta + h)
    = u D
\end{equation}

\begin{equation}
    \label{eq:swe-n}
    N =
    \int_{-h}^{\eta}
    v (x, y, z, t)
    \,dz =
    v (x, y, \tilde{z}, t)
    \int_{-h}^{\eta}
    \,dz =
    v (x, y, \tilde{z}, t)
    (\eta + h)
    = v D
\end{equation}

Esse resultado é obtido a partir da simplificação típica do modelo de águas rasas, na qual se
assume que as velocidades nas direções $x$ e $y$ variam muito pouco ao longo da profundidade.
Essa hipótese é justificada pelo fato de que, em escoamentos de águas rasas, a altura da coluna d’água
é muito menor que as dimensões horizontais do domínio, tornando razoável considerar as componentes
horizontais da velocidade aproximadamente constantes em $z$.

Além disso, é necessário especificar algumas condições de contorno para derivar as equações de águas rasas.
A primeira condição de contorno é a cinemática na superfície livre, para obter essa essa condição de contorno
é necessário definir implicitamente a equação da superfície livre, dada por $F(x, y, z, t) = z - \eta(x, y, t)$.

\begin{equation}
    \label{eq:bc1.1}
    \begin{aligned}
        &\frac{\partial F}{\partial t} + (\mathbf{u} \cdot \nabla) F = 0 \\
        &\frac{\partial F}{\partial t} + u \frac{\partial F}{\partial x}
        + v \frac{\partial F}{\partial y} + w \frac{\partial F}{\partial z} = 0
    \end{aligned}
\end{equation}

Substituindo $F$ por $z - \eta$ e resolvendo as derivdas parciais.

\begin{equation}
    \label{eq:bc1.2}
    \begin{aligned}
        &\frac{\partial (z - \eta)}{\partial t}
        + u \frac{\partial (z - \eta)}{\partial x}
        + v \frac{\partial (z - \eta)}{\partial y}
        + w \frac{\partial (z - \eta)}{\partial z} = 0\\
        &\frac{\partial \eta}{\partial t}
        + u \frac{\partial \eta}{\partial x}
        + v \frac{\partial \eta}{\partial y}
        = w
    \end{aligned}
\end{equation}

A segunda condição de contorno a ser considerada é a pressão atmosférica constante ao longo da superfície
livre.

\begin{equation}
    \label{eq:bc2}
    p \big|_{z = \eta} = p_0
\end{equation}

A terceira condição de contorno é a condição de não-deslizamento aplicada às velocidades na base da coluna d’água.

\begin{equation}
    \label{eq:bc3}
    u \big|_{z = -h} =
    v \big|_{z = -h} =
    w \big|_{z = -h} = 0
\end{equation}

Com as condições de contorno estabelecidas, podemos prosseguir com a derivação das equações. Considerando a Figura
\ref{fig:figura-swe} e integrando a primeira equação do sistema \eqref{eq:swe-sis-3} em relação à variável
$z$, obtemos:

\begin{equation}
    \label{eq:swe1}
    \int_{-h}^{\eta}
    \left(
        + \frac{\partial u}{\partial x}
        + \frac{\partial v}{\partial y}
        + \frac{\partial w}{\partial z}
    \right)
    \,dz = 0
\end{equation}

Integrando cada componente individualmente

\begin{equation}
    \label{eq:swe1.1}
    \int_{-h}^{\eta}
    \left(
        \frac{\partial u}{\partial x}
    \right)
    \,dz =
    \frac{\partial}{\partial x}
    \int_{-h}^{\eta} u \,dz -
    u \big|_{z=\eta} \frac{\partial \eta}{\partial x} -
    u \big|_{z=-h} \frac{\partial h}{\partial x}
\end{equation}

\begin{equation}
    \label{eq:swe1.2}
    \int_{-h}^{\eta}
    \left(
        \frac{\partial v}{\partial y}
    \right)
    \,dz =
    \frac{\partial}{\partial y}
    \int_{-h}^{\eta} v \,dz -
    v \big|_{z=\eta} \frac{\partial \eta}{\partial y} -
    v \big|_{z=-h} \frac{\partial h}{\partial y}
\end{equation}

\begin{equation}
    \label{eq:swe1.3}
    \int_{-h}^{\eta}
    \left(
        \frac{\partial w}{\partial z}
    \right)
    \,dz =
    w\big|_{z=\eta} - w\big|_{z=-h}
\end{equation}

Substituindo \eqref{eq:swe1.1}, \eqref{eq:swe1.2}, e \eqref{eq:swe1.3} em \eqref{eq:swe1}

\begin{equation}
    \label{eq:swe1-final1}
    \begin{aligned}
&\Biggl(
    \frac{\partial}{\partial x} \int_{-h}^{\eta} u \, dz
    - u\big|_{z=\eta} \frac{\partial \eta}{\partial x}
    - u\big|_{z=-h} \frac{\partial h}{\partial x}
\Biggr) \;+\; \\
&\Biggl(
    \frac{\partial}{\partial y} \int_{-h}^{\eta} v \, dz
    - v\big|_{z=\eta} \frac{\partial \eta}{\partial y}
    - v\big|_{z=-h} \frac{\partial h}{\partial y}
\Biggr) \;+\;
\Bigl(
    w\big|_{z=\eta} - w\big|_{z=-h}
\Bigr) = 0
    \end{aligned}
\end{equation}

Considerando a condição de contorno de não-deslizamento.

\begin{equation}
    \label{eq:swe1-final2}
    \frac{\partial (uD)}{\partial x} -
    u\big|_{z=\eta} \frac{\partial \eta}{\partial x} +
    \frac{\partial (vD)}{\partial y} -
    v\big|_{z=\eta} \frac{\partial \eta}{\partial y} +
    w\big|_{z=\eta} = 0
\end{equation}

Considerando a condição de contorno da cinemática ao longo da superfície livre e as equações \eqref{eq:swe-m} e \eqref{eq:swe-n}

\begin{equation}
    \label{eq:swe1-final3}
    \frac{\partial (uD)}{\partial x}
    - u\big|_{z=\eta} \frac{\partial \eta}{\partial x}
    + \frac{\partial (vD)}{\partial y}
    - v\big|_{z=\eta} \frac{\partial \eta}{\partial y}
    + \frac{\partial \eta}{\partial t}
    + u\big|_{z=\eta} \frac{\partial \eta}{\partial x}
    + v\big|_{z=\eta} \frac{\partial \eta}{\partial y}
    = 0
\end{equation}

\begin{equation}
    \label{eq:swe1-final4}
    \frac{\partial M}{\partial x}
    + \frac{\partial N}{\partial y}
    + \frac{\partial \eta}{\partial t}
    = 0
\end{equation}

Repetindo o processo para a segunda equação do sistema \eqref{eq:swe-sis-3}:

\begin{equation}
    \label{eq:swe2}
    \int_{-h}^{\eta}
    \left(
        \frac{\partial u}{\partial t}
        + \frac{\partial (u^2 + p)}{\partial x}
        + \frac{\partial (u v)}{\partial y}
        + \frac{\partial (u w)}{\partial z}
    \right)
    \,dz = 0
\end{equation}

Integrando cada componente individualmente

\begin{equation}
    \label{eq:swe2.1}
    \int_{-h}^{\eta}
    \left(
        \frac{\partial u}{\partial t}
    \right)
    \,dz =
    \frac{\partial}{\partial t}
    \int_{-h}^{\eta} u \,dz -
    u \big|_{z=\eta} \frac{\partial \eta}{\partial t} -
    u \big|_{z=-h} \frac{\partial h}{\partial t}
\end{equation}

\begin{equation}
    \label{eq:swe2.2}
    \int_{-h}^{\eta}
    \left(
        \frac{\partial (u^2 + p)}{\partial x}
    \right)
    \,dz =
    \frac{\partial}{\partial x}
    \int_{-h}^{\eta} (u^2 + p) \,dz -
    (u^2 + p) \big|_{z=\eta} \frac{\partial \eta}{\partial x} -
    (u^2 + p) \big|_{z=-h} \frac{\partial h}{\partial x}
\end{equation}

\begin{equation}
    \label{eq:swe2.3}
    \int_{-h}^{\eta}
    \left(
        \frac{\partial (uv)}{\partial y}
    \right)
    \,dz =
    \frac{\partial}{\partial y}
    \int_{-h}^{\eta} (uv) \,dz -
    (uv) \big|_{z=\eta} \frac{\partial \eta}{\partial y} -
    (uv) \big|_{z=-h} \frac{\partial h}{\partial y}
\end{equation}

\begin{equation}
    \label{eq:swe2.4}
    \int_{-h}^{\eta}
    \left(
        \frac{\partial (u w)}{\partial z}
    \right)
    \,dz =
    (u w)\big|_{z=\eta} - (u w)\big|_{z=-h}
\end{equation}

Substituindo \eqref{eq:swe2.1}, \eqref{eq:swe2.2}, \eqref{eq:swe2.3} e \eqref{eq:swe2.4} em \eqref{eq:swe2}, obtém-se:

\begin{equation}
\label{eq:swe2-final1}
\begin{aligned}
    &\Biggl(
        \frac{\partial}{\partial t} \int_{-h}^{\eta} u \,dz
        - u\big|_{z=\eta} \frac{\partial \eta}{\partial t}
        - u\big|_{z=-h} \frac{\partial h}{\partial t}
    \Biggr)
    \\
    &+
    \Biggl(
        \frac{\partial}{\partial x} \int_{-h}^{\eta} (u^2 + p) \,dz
        - (u^2 + p)\big|_{z=\eta} \frac{\partial \eta}{\partial x}
        - (u^2 + p)\big|_{z=-h} \frac{\partial h}{\partial x}
    \Biggr)
    \\
    &+
    \Biggl(
        \frac{\partial}{\partial y} \int_{-h}^{\eta} (u v) \,dz
        - (u v)\big|_{z=\eta} \frac{\partial \eta}{\partial y}
        - (u v)\big|_{z=-h} \frac{\partial h}{\partial y}
    \Biggr)
    \\
    &+
    \Biggl(
        (u w)\big|_{z=\eta} - (u w)\big|_{z=-h}
    \Biggr) = 0
\end{aligned}
\end{equation}

Considerando a condição de contorno de não-deslizamento.

\begin{equation}
\label{eq:swe2-final2}
\begin{aligned}
    &\frac{\partial (u D)}{\partial t} 
    - u\big|_{z=\eta} \frac{\partial \eta}{\partial t}
    + \frac{\partial}{\partial x} 
      \int_{-h}^{\eta} (u^2 + p_0 - g z + g \eta) \,dz 
    - u^2 \big|_{z=\eta} \frac{\partial \eta}{\partial x} 
    - p_0 \frac{\partial \eta}{\partial x} \\
    &- (p_0 + g D) \frac{\partial h}{\partial x}
    + \frac{\partial (u v D)}{\partial y} 
    - (u v)\big|_{z=\eta} \frac{\partial \eta}{\partial y} 
    + (u w)\big|_{z=\eta} = 0
\end{aligned}
\end{equation}

\begin{equation}
\label{eq:swe2-final3}
\begin{aligned}
    &\frac{\partial (u D)}{\partial t} 
    + \frac{\partial}{\partial x} 
      \Bigl(
          u^2 D + p_0 D - \frac{g}{2}(\eta^2 - h^2) + g \eta D
      \Bigr)
    + \frac{\partial (u v D)}{\partial y} \\
    &- u\big|_{z=\eta} 
      \Bigl(
          \frac{\partial \eta}{\partial t} 
          + u\big|_{z=\eta} \frac{\partial \eta}{\partial x} 
          + v\big|_{z=\eta} \frac{\partial \eta}{\partial y} 
          - w\big|_{z=\eta}
      \Bigr)
      - p_0 \frac{\partial \eta}{\partial x} 
      - p_0 \frac{\partial h}{\partial x} 
      - g D \frac{\partial h}{\partial x} = 0
\end{aligned}
\end{equation}

Considerando a condição de contorno da cinemática ao longo da superfície livre e as equações \eqref{eq:swe-m} e \eqref{eq:swe-n}

\begin{equation}
\label{eq:swe2-final4}
    \frac{\partial M}{\partial t} 
    + \frac{\partial }{\partial x} \left( \frac{M^2}{D} \right)
    + \frac{\partial}{\partial x} 
      \Bigl(
          p_0 D - \frac{g}{2}(D^2 + 2 D h) + g (D - h) D
      \Bigr)
      + \frac{\partial}{\partial y} \left( \frac{M N}{D}\right)
    - p_0 \frac{\partial D}{\partial x} 
    - g D \frac{\partial h}{\partial x} = 0
\end{equation}

\begin{equation}
\label{eq:swe2-final5}
    \frac{\partial M}{\partial t} 
    + \frac{\partial }{\partial x} \left( \frac{M^2}{D} \right)
    + \frac{\partial}{\partial x} 
      \Bigl(
          - \frac{g}{2}(D^2 + 2 D h) + g D^2 - g D h
      \Bigr)
      + \frac{\partial}{\partial y} \left( \frac{M N}{D}\right)
    + p_0 \frac{\partial D}{\partial x} 
    - p_0 \frac{\partial D}{\partial x} 
    - g D \frac{\partial h}{\partial x} = 0
\end{equation}

\begin{equation}
\label{eq:swe2-final6}
    \frac{\partial M}{\partial t} 
    + \frac{\partial }{\partial x} \left( \frac{M^2}{D} \right)
    + \frac{\partial}{\partial x} 
      \Bigl(
          \frac{g D^2}{2} - g D h + g D h
      \Bigr)
      + \frac{\partial}{\partial y} \left( \frac{M N}{D}\right)
    - g D \frac{\partial h}{\partial x} = 0
\end{equation}

\begin{equation}
\label{eq:swe2-final7}
    \frac{\partial M}{\partial t} 
    + \frac{\partial }{\partial x} \left( \frac{M^2}{D} \right)
    + \frac{\partial}{\partial y} \left( \frac{M N}{D}\right)
    + \frac{g}{2} \frac{\partial D^2}{\partial x} 
    - g D \frac{\partial h}{\partial x} = 0
\end{equation}

\begin{equation}
\label{eq:swe2-final8}
    \frac{\partial M}{\partial t} 
    + \frac{\partial }{\partial x} \left( \frac{M^2}{D} \right)
    + \frac{\partial}{\partial y} \left( \frac{M N}{D}\right)
    + g D \frac{\partial D}{\partial x} 
    - g D \frac{\partial h}{\partial x} = 0
\end{equation}

\begin{equation}
\label{eq:swe2-final9}
    \frac{\partial M}{\partial t} 
    + \frac{\partial }{\partial x} \left( \frac{M^2}{D} \right)
    + \frac{\partial}{\partial y} \left( \frac{M N}{D}\right)
    + g D \frac{\partial \eta}{\partial x} = 0
\end{equation}

Repetindo o processo para a terceira equação do sistema \eqref{eq:swe-sis-3}:

\begin{equation}
    \label{eq:swe3}
    \int_{-h}^{\eta}
    \left(
        \frac{\partial v}{\partial t}
        + \frac{\partial (u v)}{\partial x}
        + \frac{\partial (v^2 + p)}{\partial y}
        + \frac{\partial (v w)}{\partial z}
    \right)
    \,dz = 0
\end{equation}

Integrando cada componente individualmente

\begin{equation}
    \label{eq:swe3.1}
    \int_{-h}^{\eta}
    \left(
        \frac{\partial v}{\partial t}
    \right)
    \,dz =
    \frac{\partial}{\partial t}
    \int_{-h}^{\eta} v \,dz -
    v \big|_{z=\eta} \frac{\partial \eta}{\partial t} -
    v \big|_{z=-h} \frac{\partial h}{\partial t}
\end{equation}

\begin{equation}
    \label{eq:swe3.2}
    \int_{-h}^{\eta}
    \left(
        \frac{\partial (uv)}{\partial x}
    \right)
    \,dz =
    \frac{\partial}{\partial x}
    \int_{-h}^{\eta} (uv) \,dz -
    (uv) \big|_{z=\eta} \frac{\partial \eta}{\partial x} -
    (uv) \big|_{z=-h} \frac{\partial h}{\partial x}
\end{equation}

\begin{equation}
    \label{eq:swe3.3}
    \int_{-h}^{\eta}
    \left(
        \frac{\partial (v^2 + p)}{\partial y}
    \right)
    \,dz =
    \frac{\partial}{\partial y}
    \int_{-h}^{\eta} (v^2 + p) \,dz -
    (v^2 + p) \big|_{z=\eta} \frac{\partial \eta}{\partial y} -
    (v^2 + p) \big|_{z=-h} \frac{\partial h}{\partial y}
\end{equation}

\begin{equation}
    \label{eq:swe3.4}
    \int_{-h}^{\eta}
    \left(
        \frac{\partial (v w)}{\partial z}
    \right)
    \,dz =
    (v w)\big|_{z=\eta} - (v w)\big|_{z=-h}
\end{equation}

Substituindo \eqref{eq:swe3.1}, \eqref{eq:swe3.2}, \eqref{eq:swe3.3} e \eqref{eq:swe3.4} em \eqref{eq:swe3}, obtém-se:

\begin{equation}
    \label{eq:swe3-final1}
    \begin{aligned}
        &\Bigl(
            \frac{\partial}{\partial t} \int_{-h}^{\eta} v \, dz
            - v \big|_{z=\eta} \frac{\partial \eta}{\partial t}
            - v \big|_{z=-h} \frac{\partial h}{\partial t}
        \Bigr) \\
        &+ \Bigl(
            \frac{\partial}{\partial x} \int_{-h}^{\eta} (uv) \, dz
            - (uv) \big|_{z=\eta} \frac{\partial \eta}{\partial x}
            - (uv) \big|_{z=-h} \frac{\partial h}{\partial x}
        \Bigr) \\
        &+ \Bigl(
            \frac{\partial}{\partial y} \int_{-h}^{\eta} (v^2 + p) \, dz
            - (v^2 + p) \big|_{z=\eta} \frac{\partial \eta}{\partial y}
            - (v^2 + p) \big|_{z=-h} \frac{\partial h}{\partial y}
        \Bigr) \\
        &+ \Bigl(
            (v w) \big|_{z=\eta} - (v w) \big|_{z=-h}
        \Bigr) = 0
    \end{aligned}
\end{equation}

Considerando a condição de contorno de não-deslizamento.

\begin{equation}
    \label{eq:swe3-final2}
    \begin{aligned}
        &\Bigl(
            \frac{\partial (vD)}{\partial t}
            - v \big|_{z=\eta} \frac{\partial \eta}{\partial t}
        \Bigr)
        + \Bigl(
            \frac{\partial (uvD)}{\partial x}
            - (uv) \big|_{z=\eta} \frac{\partial \eta}{\partial x}
        \Bigr) \\
        &+ \Bigl(
            \frac{\partial}{\partial y} \int_{-h}^{\eta} (v^2 + p_0 - g z + g \eta) \, dz
            - v^2 \big|_{z=\eta} \frac{\partial \eta}{\partial y}
            - p_0 \frac{\partial \eta}{\partial y}
            - (p_0 + g D) \frac{\partial h}{\partial y}
        \Bigr)
        + \Bigl(
            (v w) \big|_{z=\eta}
        \Bigr) = 0
    \end{aligned}
\end{equation}

\begin{equation}
    \label{eq:swe3-final3}
    \begin{aligned}
        &\Bigl(
            \frac{\partial (vD)}{\partial t}
        \Bigr)
        + \Bigl(
            \frac{\partial (uvD)}{\partial x}
        \Bigr)
        + \Bigl(
            \frac{\partial}{\partial y} \int_{-h}^{\eta} (v^2 + p_0 - g z + g \eta) \, dz
            - p_0 \frac{\partial \eta}{\partial y}
            - (p_0 + g D) \frac{\partial h}{\partial y}
        \Bigr) \\
        &- v \big|_{z=\eta}
        \left(
            \frac{\partial \eta}{\partial t}
            + u \frac{\partial \eta}{\partial x}
            + v \frac{\partial \eta}{\partial y}
            + w
        \right) = 0
    \end{aligned}
\end{equation}

Considerando a condição de contorno da cinemática ao longo da superfície livre e as equações \eqref{eq:swe-m} e \eqref{eq:swe-n}

\begin{equation}
\label{eq:swe3-final4}
    \frac{\partial N}{\partial t}
    + \frac{\partial }{\partial x} \left(\frac{MN}{D}\right)
    + \frac{\partial}{\partial y} \left(v^2 D + p_0 D -\frac{g}{2} (\eta^2 - h^2) + g \eta D\right)
    - p_0 \frac{\partial \eta}{\partial y}
    - p_0 \frac{\partial h}{\partial y}
    - g D \frac{\partial h}{\partial y} = 0
\end{equation}

\begin{equation}
\label{eq:swe3-final5}
    \frac{\partial N}{\partial t}
    + \frac{\partial }{\partial x} \left(\frac{MN}{D}\right)
    + \frac{\partial }{\partial y} \left(\frac{N^2}{D}\right)
    + \frac{\partial}{\partial y} \left(-\frac{g}{2} (D^2 + 2 D h) + g (D - h) D\right)
    + p_0 \frac{\partial D}{\partial y}
    - p_0 \frac{\partial D}{\partial y}
    - g D \frac{\partial h}{\partial y} = 0
\end{equation}

\begin{equation}
\label{eq:swe3-final6}
    \frac{\partial N}{\partial t}
    + \frac{\partial }{\partial x} \left(\frac{MN}{D}\right)
    + \frac{\partial }{\partial y} \left(\frac{N^2}{D}\right)
    + \frac{\partial}{\partial y} \left(-\frac{g}{2} (D^2 + 2 D h) + g D^2 - g D h\right)
    - g D \frac{\partial h}{\partial y} = 0
\end{equation}

\begin{equation}
\label{eq:swe3-final7}
    \frac{\partial N}{\partial t}
    + \frac{\partial }{\partial x} \left(\frac{MN}{D}\right)
    + \frac{\partial }{\partial y} \left(\frac{N^2}{D}\right)
    + \frac{\partial}{\partial y} \left(\frac{g D^2}{2} + g D h - g D h\right)
    - g D \frac{\partial h}{\partial y} = 0
\end{equation}

\begin{equation}
\label{eq:swe3-final8}
    \frac{\partial N}{\partial t}
    + \frac{\partial }{\partial x} \left(\frac{MN}{D}\right)
    + \frac{\partial }{\partial y} \left(\frac{N^2}{D}\right)
    +  \frac{g}{2} \frac{\partial D^2}{\partial y}
    - g D \frac{\partial h}{\partial y} = 0
\end{equation}

\begin{equation}
\label{eq:swe3-final9}
    \frac{\partial N}{\partial t}
    + \frac{\partial }{\partial x} \left(\frac{MN}{D}\right)
    + \frac{\partial }{\partial y} \left(\frac{N^2}{D}\right)
    + g D \frac{\partial D}{\partial y}
    - g D \frac{\partial h}{\partial y} = 0
\end{equation}

\begin{equation}
\label{eq:swe3-final10}
    \frac{\partial N}{\partial t}
    + \frac{\partial }{\partial x} \left(\frac{MN}{D}\right)
    + \frac{\partial }{\partial y} \left(\frac{N^2}{D}\right)
    + g D \frac{\partial \eta}{\partial y} = 0
\end{equation}

Combinando as equações \eqref{eq:swe1-final4}, \eqref{eq:swe2-final9} e \eqref{eq:swe3-final8} em um sistema de equações

\[
\begin{cases}
\displaystyle
\frac{\partial M}{\partial x}
+ \frac{\partial N}{\partial y}
+ \frac{\partial \eta}{\partial t}
= 0,
\\[8pt]
\displaystyle
\frac{\partial M}{\partial t}
+ \frac{\partial}{\partial x}\!\left(\frac{M^{2}}{D}\right)
+ \frac{\partial}{\partial y}\!\left(\frac{M N}{D}\right)
+ g\,D\,\frac{\partial \eta}{\partial x}
= 0,
\\[8pt]
\displaystyle
\frac{\partial N}{\partial t}
+ \frac{\partial}{\partial x}\!\left(\frac{M N}{D}\right)
+ \frac{\partial}{\partial y}\!\left(\frac{N^{2}}{D}\right)
+ g\,D\,\frac{\partial \eta}{\partial y}
= 0.
\end{cases}
\]


\subsection{Discretização}
De acordo com \textcite{cuminato2013discretizacao}, a aproximação numérica da solução de uma equação de derivadas parciais é obtida por meio
da transformação do problema contínuo num problema discreto e finito. Como no caso de equações diferenciais parcias não temos apenas uma variável
independente, o domínio passa a ser uma região do plano ou do espaço. O domínio é transformado em um conjunto de pontos, geralmente igualmente espaçados,
denominado \emph{malha}. As equações passam a ser avaliadas apenas nos pontos dessa malha e as derivadas são aproximadas por
diferenças finitas.

\begin{figure}[H]
    \centering
    \caption{Exemplo de malha}
    \includegraphics[width=0.5\linewidth]{images/grid.jpg}\\
    \fonteimagem{Fonte: \citeauthor{venkatratnam2015}, \citeyear{venkatratnam2015}}
    \label{fig:figura-grid}
\end{figure}

\subsection{Diferenças finitas}
O método das difereças finitas é um método numérico para se calcular derivadas de funções, essencial quando não se tem uma função geradora dos dados do
problema para se calcular de forma analítica, consiste em aproximar a derivada em um ponto usando um conjunto de vizinhos, denominamos esse conjunto de
vizinhos de estêncil.

As aproximações mais utilizadas para a derivada de primeira ordem são as diferenças progressiva (explícita), regressiva (implícita) e centrada.

\subsubsection{Diferença progressiva}

\[
    f'(x) = \frac{f(x+\Delta x) - f(x)}{\Delta x}
\]

\subsubsection{Diferença regressiva}

\[
    f'(x) = \frac{f(x) - f(x-\Delta x)}{\Delta x}
\]

\subsubsection{Diferença centrada}

\[
    f'(x) = \frac{f(x+\Delta x) - f(x-\Delta x)}{2 \Delta x}
\]

A diferença centrada apresenta, em geral maior ordem de precisão em comparação com as outras duas, a diferença centrada possui erro de ordem $O(\Delta x^2)$
em quanto que as outras duas apresentam erro de ordem $O(\Delta x)$

\newpage
